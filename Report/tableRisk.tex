\newcommand{\tableRiskManager}[0]
{ 
    \getimg{Bản kế hoạch đơn giản cho dự án:}{}{Ảnh Chương 1/Bản kế hoạch.png}{}
    \textit{Bản quản lý các rủi do đơn giản trong quá trình thực hiện dự án:}
    
    \begin{table}[H]
        \centering
        \arrayrulecolor{black} % Màu đường kẻ ngăn cách
        \resizebox{\textwidth}{!}
        {
            \begin{tabular}{|m{1.5cm}|m{2cm}|m{2cm}|m{2cm}|m{2cm}|m{2cm}|}
                \hline
                \rule[-1cm]{}{2cm} \multirow{2}{\parbox{1.5cm}{\RaggedRight Công việc / Hoạt động}} & 
                \multicolumn{3}{c|}{Rủi ro} & 
                \multicolumn{2}{c|}{Quản lý rủi ro}
                \\
                \cline{2-6}
                 & \makecell{Mối nguy} & \makecell{Rủi ro} & \makecell{Mức độ} & \makecell{Chiến lược} & \makecell{Biện pháp} \\ 
                \hline
                Thống kê, ghi & Bị mất dữ liệu & {\setlength{\spaceskip}{0.2em}Không có dữ liệu phòng bị} & Trung bình & Phòng tránh & Sao lưu dữ liệu \\
                \hline
                Nhập số tiền nộp & {\setlength{\spaceskip}{0.2em}Nhập sai dữ liệu} & & Thấp & & {\setlength{\spaceskip}{0.5em}Thường xuyên kiểm tra lại} \\
                \hline
            \end{tabular}
        }
    \end{table}
}