\documentclass{article}
\usepackage{booktabs}
\usepackage[utf8]{inputenc}
\usepackage[T5]{fontenc}
\usepackage[fontsize=13pt]{scrextend}
\usepackage[paperheight=29.7cm, paperwidth=21cm, right=2cm, left=3cm, top=2cm, bottom=2.5cm]{geometry}
\usepackage{mathptmx}
\usepackage{graphicx} % Gói dùng để chèn ảnh
\usepackage{caption}  % Gói để tùy chỉnh caption
\usepackage{float}
\usepackage{tikz}
\usetikzlibrary{calc}
\usepackage{indentfirst} % Thư viện thụt đầu dòng
\usepackage{setspace}
\renewcommand{\baselinestretch}{1.2} % Dãn dòng
\setlength{\parskip}{6pt}
% \setlength{\parindent}{1cm}
\usepackage{titlesec} % Thư viện set up các kiểu chữ
\usepackage{tcolorbox}
\usepackage{tocloft} % Gói để tuỳ chỉnh mục lục
\usepackage{changepage} % Gói adjustwidth
\usepackage{makecell}
\usepackage{tabularx}
\usepackage{xcolor}
\usepackage[table]{xcolor}
\usepackage{hyperref}    % Liên kết trong mục lục
\usepackage{multirow}
\usepackage{ragged2e}
\usepackage{pgfplotstable}

\usepackage{fancyhdr} % Import gói fancyhdr
\setcounter{secnumdepth}{4}
\usepackage{enumitem}

\renewcommand{\headrulewidth}{0.4pt}       % Đường kẻ header
\renewcommand{\footrulewidth}{0.4pt}       % Đường kẻ footer
\renewcommand{\cftsecleader}{\cftdotfill{\cftdotsep}} % Thêm dấu chấm trong mục lục

\renewcommand{\contentsname}{\textbf{MỤC LỤC}} % Đặt tên mục lục
\renewcommand{\cfttoctitlefont}{\hfill\fontsize{16pt}{0pt}\selectfont\bfseries} % Căn giữa và đổi font chữ
\renewcommand{\cftaftertoctitle}{\hfill} % Giữ căn giữa sau tiêu đề

\pagestyle{fancy}
\fancyhf{} % Xóa header và footer mặc định
\fancyhead[L]{\textbf{IT3180 - Nhập môn công nghệ phần mềm}} % Header góc trái
\fancyfoot[L]{Nhóm 1} % Footer góc trái
\fancyfoot[R]{\thepage} % Footer góc phải

\titlespacing*{\section}{0pt}{0pt}{30pt} % Heading 1
\titleformat*{\section}{\fontsize{16pt}{0pt}\selectfont \bfseries \centering}

\titlespacing*{\subsection}{0pt}{10pt}{0pt} % Heading 2
\titleformat*{\subsection}{\fontsize{14pt}{0pt}\selectfont \bfseries}

\titlespacing*{\subsubsection}{0pt}{10pt}{0pt} % Heading 3
\titleformat*{\subsubsection}{\fontsize{13pt}{0pt}\selectfont \bfseries \itshape}

\titlespacing*{\paragraph}{0pt}{10pt}{0pt} % Heading 4
\titleformat*{\paragraph}{\fontsize{13pt}{0pt}\selectfont \itshape}

\input{border}
\newcommand{\getimg}[4]
{
    \textit{#1}
    \begin{figure}[H]
        \centering
        \includegraphics[width=#2\linewidth]{#3}
        \caption*{#4}
    \end{figure}
}
\newcommand{\tableRiskManager}[0]
{ 
    \getimg{Bản kế hoạch đơn giản cho dự án:}{}{Ảnh Chương 1/Bản kế hoạch.png}{}
    \textit{Bản quản lý các rủi do đơn giản trong quá trình thực hiện dự án:}
    
    \begin{table}[H]
        \centering
        \arrayrulecolor{black} % Màu đường kẻ ngăn cách
        \resizebox{\textwidth}{!}
        {
            \begin{tabular}{|m{1.5cm}|m{2cm}|m{2cm}|m{2cm}|m{2cm}|m{2cm}|}
                \hline
                \rule[-1cm]{}{2cm} \multirow{2}{\parbox{1.5cm}{\RaggedRight Công việc / Hoạt động}} & 
                \multicolumn{3}{c|}{Rủi ro} & 
                \multicolumn{2}{c|}{Quản lý rủi ro}
                \\
                \cline{2-6}
                 & \makecell{Mối nguy} & \makecell{Rủi ro} & \makecell{Mức độ} & \makecell{Chiến lược} & \makecell{Biện pháp} \\ 
                \hline
                Thống kê, ghi & Bị mất dữ liệu & {\setlength{\spaceskip}{0.2em}Không có dữ liệu phòng bị} & Trung bình & Phòng tránh & Sao lưu dữ liệu \\
                \hline
                Nhập số tiền nộp & {\setlength{\spaceskip}{0.2em}Nhập sai dữ liệu} & & Thấp & & {\setlength{\spaceskip}{0.5em}Thường xuyên kiểm tra lại} \\
                \hline
            \end{tabular}
        }
    \end{table}
}
\newcommand{\feeInfo}[1]
{
    \renewcommand{\arraystretch}{1.5} % Tăng khoảng cách giữa các dòng trong bảng
    \begin{table}[!h]
    \centering
    \arrayrulecolor{#1} % Màu đường kẻ ngăn cách
    \begin{tabular}{|c|p{3cm}|p{5cm}|p{4cm}|}
        \hline
        \multicolumn{2}{|c|}{\cellcolor{lightaqua}\textbf{Input}} & 
        \cellcolor{lightaqua}\makecell[c]{\textbf{Process}} & 
        \cellcolor{lightaqua}\makecell[c]{\textbf{Output}}  \\
        \hline
        \multirow{5}{*}{Phí bắt buộc} & Số hộ gia đình & \multirow{5}{*}{{\parbox{5cm}{\RaggedRight Tính toán số tiền nộp của mỗi hộ gia đình}}} & 
        \multirow{11}{*}{\parbox{4cm}{\RaggedRight Số tiền mà mỗi hộ gia đình đã nộp.Tổng số tiền cả khu phố. Số hộ gia đình chưa nộp phí. Số loại phí mà mỗi hộ gia đình đã nộp. Số tiền còn nợ. (Các nghiệp vụ liên quan như: thêm, sửa, xóa, thống kê,in ấn giấy tờ,…)}}
        \\
        \cline{2-2}
         & Địa chỉ & & \\
        \cline{2-2}
         & Hộ tên chủ hộ & & \\
        \cline{2-2} 
         & Số nhân khẩu & & \\
        \cline{1-3}
         & Ngày nộp & & \\
        \cline{2-2}
        \multirow{5}{*}{Phí tự nguyện} & Số hộ gia đình & \multirow{5}{*}{\parbox{5cm}{\RaggedRight Tổng số tiền thu được trong từng đợt. Số hộ nộp tiền. Thống kê danh sách (sắp xếp theo số tiền, số lần nộp,..)}} & \\
        \cline{2-2} 
        & Địa chỉ & & \\
        \cline{2-2} 
        & Hộ tên chủ hộ & & \\
        \cline{2-2} 
        & {\setlength{\spaceskip}{0.5em}Đợt nộp (Từ thiện, ủng hộ lũ lụt, khuyến học, ...)} & & \\
        \cline{2-2}  
        & Số tiền & & \\
        \cline{2-2} 
        & Ngày nộp & & \\
        \hline   
    \end{tabular}
\end{table}
}

\newcommand{\BFDInfo}[0]
{
    \begin{table}[!h]
        \centering
        \resizebox{\textwidth}{!}
        {
            \begin{tabular}{|m{3cm}|m{5cm}|m{5cm}|}
                \hline
                \rowcolor{lightaqua}
                \makecell[c]{\textbf{Tên chức năng}} & 
                \makecell[c]{\textbf{Mô tả}} & 
                \setlength{\spaceskip}{0.3em}\textbf{Đánh giá khả năng thực hiện (nhân lực, thời gian, công nghệ, môi trường)} \\
                \hline
                \textbf{Lập danh sách} & Lập danh sách các hộ gia đình, danh sách thu tiền đóng góp tự nguyện, bắt buộc,... & Cao \\
                \hline
                \textbf{Thống kê} & Thống kê số tiền thu, tổng số tiền thu, tổng số hộ nộp,... & Cao \\
                \hline
                \textbf{Tra cứu} & Hỗ trợ tra cứu thông tin về danh sách nộp tiền dễ dàng hơn & Cao \\
                \hline
            \end{tabular}
        }
    \end{table}
}
\definecolor{lightblue}{HTML}{b3c5e6}
\definecolor{blueL}{HTML}{4371c3}
\newcommand{\setwhite}{\textcolor{white}}

\newcommand{\describetable}[2]
{
    \textit{#1}
    \begin{center}
        \begin{tabular}{|c|p{2.5cm}|p{2cm}|p{2cm}|p{2.5cm}|p{1.5cm}|}
            \hline
            \cellcolor{blueL}{\setwhite{Tên trường}} & 
            \cellcolor{blueL}{\setwhite{\makecell[c]{Kiểu dữ liệu}}} & \cellcolor{blueL}{\setwhite{Kích thước}} & 
            \cellcolor{blueL}{\setwhite{Ràng buộc}} \newline 
            \cellcolor{blueL}{\setwhite{toàn vẹn}} & 
            \cellcolor{blueL}{\setwhite{\makecell[c]{Khuôn dạng}}} & \cellcolor{blueL}{\setwhite{\makecell[c]{Ghi chú}}} \\
            \hline
            #2 
        \end{tabular}
    \end{center}
}

\definecolor{lightaqua}{HTML}{4aabc5}

\makeatletter
\let\ps@plain\ps@fancy
\makeatother

\begin{document}
% Bìa ---------------------------------------------------------------------------
\begin{titlepage}
\begin{Border}
    \begin{center}
        \vspace{-12pt} TRƯỜNG ĐẠI HỌC BÁCH KHOA HÀ NỘI \\
        \textbf{VIỆN CÔNG NGHỆ THÔNG TIN VÀ TRUYỀN THÔNG} \\
        \rule{0.8\textwidth}{0.5pt} \\[1cm]

        \begin{figure}[H]
            \centering
            \includegraphics[width=1.53cm, height = 2.26cm]{image/Logo_Bách_Khoa.png}
        \end{figure}
        
        \textbf{\fontsize{24pt}{0pt}\selectfont BÀI TẬP LỚN} \\[0.5cm]
        \textbf{\fontsize{14pt}{0pt}\selectfont MÔN: NHẬP MÔN CÔNG NGHỆ PHẦN MỀM} \\[1cm]
        
        {\fontsize{20pt}{0pt} \selectfont \textbf{Quản lý thu phí, đóng góp}} \\[1cm]
        
        \begin{tabular}{ll}
        \textbf{\fontsize{14pt}{0pt}\selectfont Nhóm} & \fontsize{14pt}{0pt}\selectfont : 1 \\
        \textbf{\fontsize{14pt}{0pt}\selectfont Mã lớp học} & \fontsize{14pt}{0pt}\selectfont : 154019 \\
        \textbf{\fontsize{14pt}{0pt}\selectfont Giáo viên hướng dẫn} & \fontsize{14pt}{0pt}\selectfont : ThS. Nguyễn Mạnh Tuấn \\
        \end{tabular} \\[1.5cm]
        
        \textbf{\fontsize{14pt}{0pt}\selectfont Danh sách sinh viên thực hiện:} \\[1cm]
        
        \begin{tabular}{|c|c|c|c|c|}
            \hline
            \textbf{STT} & \textbf{Họ tên} & \textbf{Mã sinh viên} & \textbf{Email} & \textbf{Lớp} \\ 
            \hline
            1 & Trương Minh Ngọc & 20220038 & ngoc.tm220038 & KTMT-01 \\ 
            \hline
            2 & Nguyễn Tùng Dương & 20224968 & duong.nt224968 & KTMT-01 \\ 
            \hline
            3 & Lê Huy Dũng & 20224960 & dung.lh224960 & KTMT-01 \\
            \hline
            4 & Nguyễn Khánh Huyền & 20225016 & huyen.nk225016 & KTMT-06 \\ 
            \hline
            5 & Đỗ Minh Phúc & 20225064 & phuc.dm225064 & KTMT-06 \\
            \hline
        \end{tabular}  

        \vspace{1.5cm}
        \fontsize{14pt}{0pt}\selectfont \textbf{Hà Nội, tháng 12 năm 2024}
    \end{center}
\end{Border}
\end{titlepage}
\newpage

% Trang mục lục ---------------------------------------------------------------------------

\captionsetup{labelformat=empty} % Xóa định dạng nhãn "Figure x:"
\phantomsection % Đảm bảo liên kết chính xác
\addcontentsline{toc}{section}{MỤC LỤC} % Thêm mục lục vào bảng mục lục
\tableofcontents % tạo mục lục tự động
\cleardoublepage

% Lời nói đầu ---------------------------------------------------------------------------
\section*{LỜI NÓI ĐẦU}
\phantomsection % Đảm bảo liên kết chính xác
\addcontentsline{toc}{section}{LỜI NÓI ĐẦU}
Quản lý thu chi là việc mà bất cứ khu phố, tổ dân phố,… đều phải giải quyết để
giúp minh bạch thông tin, công khai các khoản thu, ghi chép và lưu trữ lại những
thông tin nộp phí. Để giải quyết vấn đề này cần một phần mềm có thể thay thế hoàn
toàn những cuốn sổ ghi tay để giúp ghi lại thông tin nộp phí từ người dân, tính toán
khoản thu. Đề tài sẽ mô tả chi tiết về những bước xây dựng lên 1 phần mềm hỗ trợ
quản lý thu phí 

Để tiếp cận và hoàn thiện đề tài, nhóm em sử dụng HTML và CSS, Java Spring 
Boot để xây dựng phần mềm UI trên Desktop hỗ trợ việc quản lý thu phí. Để quản lý thu phí được hiệu quả phần mềm cần hỗ trợ việc quản lý nhân khẩu, hộ khẩu và các khoản thu. Phần
mềm xây dựng giúp thống kê các khoản nộp tiền, quản lý thông tin nhân khẩu, hộ khẩu, khoản thu và các khoản nộp.
\newpage

% Trang 2-----------------------------------------------------------

\section*{PHÂN CÔNG THÀNH VIÊN TRONG NHÓM}
\phantomsection % Đảm bảo liên kết chính xác    
\addcontentsline{toc}{section}{PHÂN CÔNG THÀNH VIÊN TRONG NHÓM}
\begin{adjustwidth}{0cm}{1cm} % Lùi vào trái và phải 1cm
\resizebox{\textwidth}{!}
{ % Đặt bảng vừa chiều rộng văn bản
\begin{tabular}{|p{2.5cm}|p{3.5cm}|p{3cm}|p{4.5cm}|p{2.5cm}|}
\hline
\textbf{Họ và tên} & \textbf{Email} & \textbf{Điện thoại} & \textbf{Công việc thực hiện} & \textbf{Đánh giá} \\ 
\hline
Trương Minh Ngọc & ngoc.tm220038 & \textbf{0866137678} & Thiết kế cơ sở dữ liệu, Hỗ trợ, đóng góp xây dựng ý tưởng và tham gia làm báo cáo & Hoàn thành \\ 
\hline
Nguyễn Tùng Dương & duong.nt224968 & \textbf{0357981704} & Đóng góp ý tưởng xây dựng, code backend, \newline tham gia làm báo cáo & Hoàn thành \\ 
\hline
Nguyễn Khánh Huyền & huyen.nk225016 & \textbf{0972950285} & Tham gia làm báo cáo, Thực hiện quá trình kiểm thử và sửa chữa lỗi code & Hoàn thành \\ 
\hline
Lê Huy Dũng & dung.lh224960 & \textbf{0974187142} & Đóng góp ý tưởng xây dựng, làm frontend giao diện của user, tham gia làm báo cáo & Hoàn thành \\ 
\hline
Đỗ Minh \newline Phúc & phuc.dm225064 & \textbf{0963908468} & Đóng góp ý tưởng xây dựng, làm frontend giao diện của admin, tham gia làm báo cáo & Hoàn thành \\ 
\hline
\end{tabular}
}
\end{adjustwidth}
\newpage

% Trang 3 --------------------------------------------------------------------------
\section*{CHƯƠNG 1. KHẢO SÁT BÀI TOÁN}
\phantomsection % Đảm bảo liên kết chính xác
\addcontentsline{toc}{section}{CHƯƠNG 1. KHẢO SÁT BÀI TOÁN}
\setcounter{section}{1}

\subsection{Mô tả yêu cầu bài toán}
Bài toán quản lý thu phí, đóng góp (yêu cầu nghiệp vụ số 2)
\begin{itemize}[leftmargin = 1.5cm]
    \item Hàng năm tổ dân phố thực hiện thu một số khoản phí và đóng góp của các hộ gia đình, công việc này do cán bộ kế toán phụ trách. Khoản phí vệ sinh là bắt buộc với tất cả các hộ gia đình, mỗi năm thu 1 lần với định mức 6.000 VNĐ / 1 tháng / 1 nhân khẩu.
    \item Cán bộ kế toán sẽ lập danh sách các hộ gia đình và số nhân khẩu tương ứng, sau đó đến từng nhà thu phí và ghi nhận số tiền nộp. Đối với các khoản đóng góp thì không quy định số tiền mà phụ thuộc vào từng hộ, các khoản đóng góp này được thu theo từng đợt của các cuộc vận động như: “Ủng hộ ngày thương binhliệt sỹ 27/07”, “Ủng hộ ngày tết thiếu nhi”, “Ủng hộ vì người nghèo”, Trợ giúp đồng bào bị ảnh hưởng bão lụt”,…
    \item Cán bộ kế toán cũng cần thống kê tổng số tiền đã thu trong mỗi đợt, tổng số hộ đã nộp và có thể xem chi tiết mỗi hộ đã nộp những khoản tiền nào.
\end{itemize}
\newpage

% ----------------------------------------------------------------------------

\subsection{Khảo sát bài toán}
Một số mẫu quản lý thu phí có sẵn theo yêu cầu của bài toán được thu thập:
\getimg{}{}{Ảnh Chương 1/Ảnh mẫu 1.jpg}{Danh sách thu quỹ}
\getimg{}{}{Ảnh Chương 1/Phiếu thu.jpg}{Mẫu phiếu thu}
\getimg{}{}{Ảnh Chương 1/Quyên góp từ thiện.jpg}{Mẫu quyên góp từ thiện}

\newpage

% ----------------------------------------------------------------------------

\subsection{Xác định thông tin cơ bản cho nghiệp vụ của bài toán}
Thông tin cơ bản cho nghiệp vụ bài toán:
\feeInfo{lightaqua} 
\newpage

% ----------------------------------------------------------------------------

\subsection{Xây dựng biểu đồ mô tả nghiệp vụ và phân cấp chức năng}
\getimg{Biểu đồ hoạt động mô tả nghiệp vụ cho bài toán:}{0.9}{Ảnh Chương 1/Activity Diagram.png}{}

\newpage

\getimg{Biểu đồ phân cấp chức năng (BFD) cho nghiệp vụ bài toán:}{0.9}{Ảnh Chương 1/BFD.png}{}

\textit{Mô tả các chức năng trong biểu đồ BFD:}
\BFDInfo
\newpage

% 1.5 ----------------------------------------------------------------------------
\subsection{Xây dựng kế hoạch dự án đơn giản}
\tableRiskManager
\newpage

% Trang mới -----------------------------------------------------------------------
\section*{CHƯƠNG 2. ĐẶC TẢ YÊU CẦU BÀI TOÁN}
\phantomsection % Đảm bảo liên kết chính xác
\addcontentsline{toc}{section}{CHƯƠNG 2. ĐẶC TẢ YÊU CẦU BÀI TOÁN}
\setcounter{section}{2}
\setcounter{subsection}{0}
\subsection{Giới thiệu chung}
Các tác nhân của hệ thống: 
\begin{itemize}
    \item[-] Người quản lý và cư dân là những người sử dụng hệ thống này, hệ thống được cung cấp thông tin từ nhân khẩu trong vùng quản lý
    \item[-] Người quản lý sẽ duy trì và quản trị hệ thống 
    \item[-] Cư dân sử dụng hệ thống để kiểm tra thông tin và nhận thông báo của chung cư
\end{itemize}

Bảng liệt kê các tác nhân và mô tả thông tin cho các tác nhân
\begin{figure}[H]
    \centering
    \includegraphics[width=15.53cm, height = 3cm]{Ảnh chương 2/Bảng tác nhân.png}
\end{figure}
Các Use Case cần thiết cho hệ thống và đặt mã cho các use-case
\begin{figure}[H]
    \centering
    \includegraphics[width=0.9\textwidth]{Ảnh chương 2/Bảng usecase.png}
\end{figure}
\subsection{Biểu đồ use case} 
\subsubsection{Biểu đồ use case tổng quan}

Để truy cập vào ứng dụng, quản lý và cư dân sử dụng tài khoản và mật khẩu được cấp sẵn với vai trò khác nhau. \\
Quản lý được cấp quyền chỉnh sửa thông tin liên quan như hộ khẩu, nhân khẩu, khoản thu qua các tính năng của trang web.\\
Cư dân có thể dùng tài khoản được cấp để xem thông tin và nhận thông báo từ quản lý
\begin{figure}[H]
    \centering
    \includegraphics[width=12.53cm, height = 8cm]{Ảnh chương 2/Tổng quát.jpg}
\end{figure}
% -----------------------------------------------------------------------
\subsubsection{Biểu đồ use case phân rã mức 2}
\begin{itemize}
    \item Phân rã Usecase "Đăng nhập"
    \begin{figure}[H]
        \centering
        \includegraphics{Ảnh chương 2/Đăng nhập.jpg}
    \end{figure}
\vspace{5cm}
    \item Phân rã Usecase "Quản lý hộ khẩu"
    \begin{figure}[H]
        \centering
        \includegraphics{Ảnh chương 2/Hộ khẩu.jpg}
    \end{figure}

    \item Phân rã Usecase "Quản lý nhân khẩu"
    \begin{figure}[H]
        \centering
        \includegraphics{Ảnh chương 2/Nhân khẩu.jpg}
    \end{figure}
    
    \vspace{1cm}
    \item Phân rã Usecase "Quản lý thu phí"
    \begin{figure}[H]
        \centering
        \includegraphics{Ảnh chương 2/Khoản thu.jpg}
    \end{figure}
\end{itemize}
% -------------------------------------------------------------------------
\subsection{Đặc tả use case}
\begin{itemize}
    \item Đăng nhập và Đăng kí
    \begin{figure}[H]
        \centering
        \includegraphics[width=0.8\textwidth]{Ảnh chương 2/UC01 1.png}
    \end{figure}
    \begin{figure}[H]
        \centering
        \includegraphics[width=0.8\textwidth]{Ảnh chương 2/UC01.png}
    \end{figure}
    \begin{figure}[H]
        \centering
        \includegraphics[width=0.8\textwidth]{Ảnh chương 2/UC02.png}
    \end{figure}
    \begin{figure}[H]
        \centering
        \includegraphics[width=0.8\textwidth]{Ảnh chương 2/UC03.png}
    \end{figure}
    \item Quản lý nhân khẩu
    \begin{figure}[H]
        \centering
        \includegraphics[width=0.8\textwidth]{Ảnh chương 2/UC04.png}
    \end{figure}
    \begin{figure}[H]
        \centering
        \includegraphics[width=0.8\textwidth]{Ảnh chương 2/UC05.png}
    \end{figure}
    \begin{figure}[H]
        \centering
        \includegraphics[width=0.8\textwidth]{Ảnh chương 2/UC06 1.png}
    \end{figure}
    \begin{figure}[H]
        \centering
        \includegraphics[width=0.8\textwidth]{Ảnh chương 2/UC06.png}
    \end{figure}
    \begin{figure}[H]
        \centering
        \includegraphics[width=0.8\textwidth]{Ảnh chương 2/UC07.png}
    \end{figure}
    \begin{figure}[H]
        \centering
        \includegraphics[width=0.8\textwidth]{Ảnh chương 2/UC08.png}
    \end{figure}
    \item Quản lý hộ khẩu
    \begin{figure}[H]
        \centering
        \includegraphics[width=0.8\textwidth]{Ảnh chương 2/UC09.png}
    \end{figure}
    \begin{figure}[H]
        \centering
        \includegraphics[width=0.8\textwidth]{Ảnh chương 2/UC10.png}
    \end{figure}
        \begin{figure}[H]
        \centering
        \includegraphics[width=0.8\textwidth]{Ảnh chương 2/UC11 1png.png}
    \end{figure}
    \begin{figure}[H]
        \centering
        \includegraphics[width=0.8\textwidth]{Ảnh chương 2/UC11.png}
    \end{figure}
    \begin{figure}[H]
        \centering
        \includegraphics[width=0.8\textwidth]{Ảnh chương 2/UC12.png}
    \end{figure}

    \begin{figure}[H]
        \centering
        \includegraphics[width=0.8\textwidth]{Ảnh chương 2/UC13.png}
    \end{figure}

    \item Quản lý khoản thu
    \begin{figure}[H]
        \centering
        \includegraphics[width=0.8\textwidth]{Ảnh chương 2/UC14.png}
    \end{figure}
    \begin{figure}[H]
        \centering
        \includegraphics[width=0.8\textwidth]{Ảnh chương 2/UC15.png}
    \end{figure}
    \begin{figure}[H]
        \centering
        \includegraphics[width=0.8\textwidth]{Ảnh chương 2/UC16.png}
    \end{figure}
    \item Quản lý tài khoản
    \begin{figure}[H]
        \centering
        \includegraphics[width=0.8\textwidth]{Ảnh chương 2/UC17.png}
    \end{figure}
    \item Tra cứu thông tin
    \begin{figure}[H]
        \centering
        \includegraphics[width=0.8\textwidth]{Ảnh chương 2/UC18.png}
    \end{figure}
\end{itemize}
\vspace{1cm}
% -------------------------------------------------------------------------
\subsection{Các yêu cầu phi chức năng}
Chức năng
\begin{itemize}
    \item Hỗ trợ nhiều quản lý và cư dân truy cập đồng thời
    \item Quản lý thông tin nhân khẩu, hộ khẩu, thông tin thu phí và khoản thu của các hộ dân cư
\end{itemize}

Tính dễ dùng
\begin{itemize}
    \item Giao diện nguời dùng tương thích Windows 7/ Window 10. Thân thiện.
\end{itemize}

Tính ổn định
\begin{itemize}
    \item Hệ thống phải hoạt động liên tục 24 giờ/ngày, 7 ngày/tuần
\end{itemize}

Hiệu suất
\begin{itemize}
    \item Hệ thống phải hỗ trợ đến 1000 người dùng truy xuất CSDL trung tâm đồng thời bất kỳ lúc nào, và đến 500 người dùng truy xuất các server cục bộ. 
    \item Hoàn thành các thao tác nhanh, chuyển giao diện không quá 2s
\end{itemize}

Sự hỗ trợ
\begin{itemize}
    \item Không
\end{itemize}

Các ràng buộc thiết kế
\begin{itemize}
    \item Không
\end{itemize}
\newpage
% ----------------------------------------------------------------------------
\section*{CHƯƠNG 3. PHÂN TÍCH YÊU CẦU}
\phantomsection % Đảm bảo liên kết chính xác
\addcontentsline{toc}{section}{CHƯƠNG 3. PHÂN TÍCH YÊU CẦU}
\setcounter{section}{3}
\setcounter{subsection}{0}
\subsection{Xác định các lớp phân tích}
\textbf{Nhóm usecase quản lý nhân khẩu}

    \getimg{Kết quả quá trình phân rã bước đầu của usecase “Xem danh sách nhân khẩu”:}{}{Ảnh chương 3/3.1/Admin/Nhân khẩu/Xem nhân khẩu.png}{}

    \getimg{Kết quả quá trình phân rã bước đầu của usecase “Tìm kiếm nhân khẩu”:}{}{Ảnh chương 3/3.1/Admin/Nhân khẩu/Tìm nhân khẩu.png}{}

    \getimg{Kết quả quá trình phân rã bước đầu của usecase “Xóa nhân khẩu”:}{}{Ảnh chương 3/3.1/Admin/Nhân khẩu/Xoá nhân khẩu.png}{}

    \getimg{Kết quả quá trình phân rã bước đầu của usecase “Thêm nhân khẩu”:}{}{Ảnh chương 3/3.1/Admin/Nhân khẩu/Thêm nhân khẩu.png}{}

    \getimg{Kết quả quá trình phân rã bước đầu của usecase “Sửa nhân khẩu”:}{}{Ảnh chương 3/3.1/Admin/Nhân khẩu/Sửa nhân khẩu.png}{}
\newpage
% --------------------------------------------------------------------
\textbf{Nhóm usecase quản lý hộ khẩu}

    \getimg{Kết quả quá trình phân rã bước đầu của usecase “Xem danh sách hộ khẩu”:}{}{Ảnh chương 3/3.1/Admin/Hộ khẩu/Xem hộ khẩu.png}{}

    \getimg{Kết quả quá trình phân rã bước đầu của usecase “Tìm kiếm hộ khẩu”:}{}{Ảnh chương 3/3.1/Admin/Hộ khẩu/Tìm hộ khẩu.png}{}
    
    \getimg{Kết quả quá trình phân rã bước đầu của usecase “Thêm hộ khẩu”:}{}{Ảnh chương 3/3.1/Admin/Hộ khẩu/Thêm hộ khẩu.png}{}
    
    \getimg{Kết quả quá trình phân rã bước đầu của usecase “Sửa hộ khẩu”:}{}{Ảnh chương 3/3.1/Admin/Hộ khẩu/Sửa hộ khẩu.png}{}
    
    \getimg{Kết quả quá trình phân rã bước đầu của usecase “Xóa hộ khẩu”:}{}{Ảnh chương 3/3.1/Admin/Hộ khẩu/Xoá hộ khẩu.png}{}
\newpage
% --------------------------------------------------------------------  
\textbf{Nhóm usecase quản lý khoản thu bắt buộc}

    \getimg{Kết quả quá trình phân rã bước đầu của usecase “Xem danh sách khoản thu bắt buộc”:}{}{Ảnh chương 3/3.1/Admin/Thu bắt buộc/Xem thu bắt buộc.png}{}
    
    \getimg{Kết quả quá trình phân rã bước đầu của usecase “Sửa khoản thu bắt buộc”:}{}{Ảnh chương 3/3.1/Admin/Thu bắt buộc/Sửa thu bắt buộc.png}{}
    
    \getimg{Kết quả quá trình phân rã bước đầu của usecase “Thêm khoản thu bắt buộc”:}{}{Ảnh chương 3/3.1/Admin/Thu bắt buộc/Thêm thu bắt buộc.png}{}
    
    \getimg{Kết quả quá trình phân rã bước đầu của usecase “Tìm kiếm khoản thu bắt buộc”:}{}{Ảnh chương 3/3.1/Admin/Thu bắt buộc/Tìm thu bắt buộc.png}{}

    \getimg{Kết quả quá trình phân rã bước đầu của usecase “Xoá khoản thu bắt buộc”:}{}{Ảnh chương 3/3.1/Admin/Thu bắt buộc/Xoá thu bắt buộc.png}{}
\newpage
% --------------------------------------------------------------------
\textbf{Nhóm usecase quản lý khoản thu tự nguyện}

    \getimg{Kết quả quá trình phân rã bước đầu của usecase “Xem danh sách khoản thu tự nguyện”:}{}{Ảnh chương 3/3.1/Admin/Thu tự nguyện/Xem thu tự nguyện.png}{}
    
    \getimg{Kết quả quá trình phân rã bước đầu của usecase “Sửa khoản thu tự nguyện”:}{}{Ảnh chương 3/3.1/Admin/Thu tự nguyện/Sửa thu tự nguyện.png}{}
    
    \getimg{Kết quả quá trình phân rã bước đầu của usecase “Thêm khoản thu tự nguyện”:}{}{Ảnh chương 3/3.1/Admin/Thu tự nguyện/Thêm thu tự nguyện.png}{}
    
    \getimg{Kết quả quá trình phân rã bước đầu của usecase “Tìm kiếm khoản thu tự nguyện”:}{}{Ảnh chương 3/3.1/Admin/Thu tự nguyện/Tìm thu tự nguyện.png}{}
\newpage

% --------------------------------------------------------------------
\textbf{Nhóm usecase quản lý Trông xe}

    \getimg{Kết quả quá trình phân rã bước đầu của usecase “Xem danh sách trông xe”:}{}{Ảnh chương 3/3.1/Admin/Trông xe/Xem trông xe.png}{}
    \getimg{Kết quả quá trình phân rã bước đầu của usecase “Thêm trông xe”:}{}{Ảnh chương 3/3.1/Admin/Trông xe/Thêm trông xe.png}{}
    \getimg{Kết quả quá trình phân rã bước đầu của usecase “Sửa trông xe”:}{}{Ảnh chương 3/3.1/Admin/Trông xe/Sửa trông xe.png}{}
    \getimg{Kết quả quá trình phân rã bước đầu của usecase “Xoá trông xe”:}{}{Ảnh chương 3/3.1/Admin/Trông xe/Xoá trông xe.png}{}
    \getimg{Kết quả quá trình phân rã bước đầu của usecase “Tìm kiếm trông xe”:}{}{Ảnh chương 3/3.1/Admin/Trông xe/Tìm trông xe.png}{}
    
\newpage

% --------------------------------------------------------------------
\textbf{Nhóm usecase quản lý Thông báo}

    \getimg{Kết quả quá trình phân rã bước đầu của usecase “Xem danh sách Thông báo”:}{}{Ảnh chương 3/3.1/Admin/Thông báo/Xem thông báo.png}{}
    \getimg{Kết quả quá trình phân rã bước đầu của usecase “Thêm Thông báo”:}{}{Ảnh chương 3/3.1/Admin/Thông báo/Thêm thông báo.png}{}
    \getimg{Kết quả quá trình phân rã bước đầu của usecase “Sửa Thông báo”:}{}{Ảnh chương 3/3.1/Admin/Thông báo/Sửa thông báo.png}{}
    \getimg{Kết quả quá trình phân rã bước đầu của usecase “Xoá Thông báo”:}{}{Ảnh chương 3/3.1/Admin/Thông báo/Xoá thông báo.png}{}
    \getimg{Kết quả quá trình phân rã bước đầu của usecase “Tìm kiếm Thông báo”:}{}{Ảnh chương 3/3.1/Admin/Thông báo/Tìm thông báo.png}{}
    
\newpage

%---------------------------------------------------------
\textbf{Nhóm usecase user xem các thông tin sau:}

    \getimg{Kết quả quá trình phân rã bước đầu của usecase user “Xem hộ khẩu”:}{}{Ảnh chương 3/3.1/User/User_Xem hộ khẩu.png}{}

    \getimg{Kết quả quá trình phân rã bước đầu của usecase user “Xem nhân khẩu”:}{}{Ảnh chương 3/3.1/User/User_Xem nhân khẩu.png}{}

    \getimg{Kết quả quá trình phân rã bước đầu của usecase user “Xem khoản thu bắt buộc”:}{}{Ảnh chương 3/3.1/User/User_Xem thu bắt buộc.png}{}

    \getimg{Kết quả quá trình phân rã bước đầu của usecase user “Xem khoản thu tự nguyện”:}{}{Ảnh chương 3/3.1/User/User_Xem thu tự nguyện.png}{}

    \getimg{Kết quả quá trình phân rã bước đầu của usecase user “Xem khoản phí gửi xe”:}{}{Ảnh chương 3/3.1/User/User_Xem gửi xe.png}{}

    \getimg{Kết quả quá trình phân rã bước đầu của usecase user “Xem khoản phí gửi xe”:}{}{Ảnh chương 3/3.1/User/User_Xem thông báo.png}{}

\newpage
% 3.2--------------------------------------------------------------------
\subsection{Xây dựng biểu đồ trình tự}
    \getimg{Biểu đồ trình tự cho usecase “Đăng nhập” phân bổ trách nhiệm ca sử dụng cho các đối tượng của các lớp phân tích:}{}{Ảnh chương 3/3.2/Đăng nhập.png}{}

%---------------------------------------------------------
    \getimg{Biểu đồ trình tự cho usecase “Thêm mới hộ khẩu” phân bổ trách nhiệm ca sử dụng cho các đối tượng của các lớp phân tích:}{}{Ảnh chương 3/3.2/Hộ khẩu/Thêm hộ khẩu.png}{}

    \getimg{Biểu đồ trình tự cho usecase “Xóa hộ khẩu” phân bổ trách nhiệm ca sử dụng cho các đối tượng của các lớp phân tích:}{}{Ảnh chương 3/3.2/Hộ khẩu/Xoá hộ khẩu.png}{}

    \getimg{Biểu đồ trình tự cho usecase “Cập nhật hộ khẩu” phân bổ trách nhiệm ca sử dụng cho các đối tượng của các lớp phân tích:}{}{Ảnh chương 3/3.2/Hộ khẩu/Sửa hộ khẩu.png}{}

    \getimg{Biểu đồ trình tự cho usecase “Tìm kiếm hộ khẩu” phân bổ trách nhiệm ca sử dụng cho các đối tượng của các lớp phân tích:}{}{Ảnh chương 3/3.2/Hộ khẩu/Tìm kiếm hộ khẩu.png}{}

    \getimg{Biểu đồ trình tự cho usecase “Xem hộ khẩu” phân bổ trách nhiệm ca sử dụng cho các đối tượng của các lớp phân tích:}{}{Ảnh chương 3/3.2/Hộ khẩu/Xem hộ khẩu.png}{}
\newpage
%---------------------------------------------------------

    \getimg{Biểu đồ trình tự cho usecase “Xem nhân khẩu” phân bổ trách nhiệm ca sử dụng cho các đối tượng của các lớp phân tích:}{}{Ảnh chương 3/3.2/Nhân khẩu/Xem nhân khẩu.png}{}

    \getimg{Biểu đồ trình tự cho usecase “Xóa nhân khẩu” phân bổ trách nhiệm ca sử dụng cho các đối tượng của các lớp phân tích:}{}{Ảnh chương 3/3.2/Nhân khẩu/Xoá nhân khẩu.png}{}

    \getimg{Biểu đồ trình tự cho usecase “Tìm kiếm nhân khẩu” phân bổ trách nhiệm ca sử dụng cho các đối tượng của các lớp phân tích:}{}{Ảnh chương 3/3.2/Nhân khẩu/Tìm kiếm nhân khẩu.png}{}

    \getimg{Biểu đồ trình tự cho usecase “Thêm nhân khẩu” phân bổ trách nhiệm ca sử dụng cho các đối tượng của các lớp phân tích:}{}{Ảnh chương 3/3.2/Nhân khẩu/Thêm nhân khẩu.png}{}

    \getimg{Biểu đồ trình tự cho usecase “Sửa nhân khẩu” phân bổ trách nhiệm ca sử dụng cho các đối tượng của các lớp phân tích:}{}{Ảnh chương 3/3.2/Nhân khẩu/Sửa nhân khẩu.png}{}
\newpage
%---------------------------------------------------------

    \getimg{Biểu đồ trình tự cho usecase “Xem khoản thu bắt buộc” phân bổ trách nhiệm ca sử dụng cho các đối tượng của các lớp phân tích:}{}{Ảnh chương 3/3.2/Khoản thu/Thêm khoản thu.png}{}

    \getimg{Biểu đồ trình tự cho usecase “Tìm kiếm khoản thu bắt buộc” phân bổ trách nhiệm ca sử dụng cho các đối tượng của các lớp phân tích:}{}{Ảnh chương 3/3.2/Khoản thu/Tìm kiếm khoản thu.png}{}

    \getimg{Biểu đồ trình tự cho usecase “Thêm khoản thu bắt buộc” phân bổ trách nhiệm ca sử dụng cho các đối tượng của các lớp phân tích:}{}{Ảnh chương 3/3.2/Khoản thu/Thêm khoản thu.png}{}

    \getimg{Biểu đồ trình tự cho usecase “Sửa khoản thu bắt buộc” phân bổ trách nhiệm ca sử dụng cho các đối tượng của các lớp phân tích:}{}{Ảnh chương 3/3.2/Khoản thu/Sửa khoản thu.png}{}

    \getimg{Biểu đồ trình tự cho usecase “Xoá khoản thu bắt buộc” phân bổ trách nhiệm ca sử dụng cho các đối tượng của các lớp phân tích:}{}{Ảnh chương 3/3.2/Khoản thu/Xoá khoản thu.png}{}
\newpage
%---------------------------------------------------------

    \getimg{Biểu đồ trình tự cho usecase “Xem khoản thu tự nguyện” phân bổ trách nhiệm ca sử dụng cho các đối tượng của các lớp phân tích:}{}{Ảnh chương 3/3.2/Đóng góp/Xem đóng góp.png}{}

    \getimg{Biểu đồ trình tự cho usecase “Tìm kiếm khoản thu tự nguyện” phân bổ trách nhiệm ca sử dụng cho các đối tượng của các lớp phân tích:}{}{Ảnh chương 3/3.2/Đóng góp/Tìm đóng góp.png}{}

    \getimg{Biểu đồ trình tự cho usecase “Thêm khoản thu tự nguyện” phân bổ trách nhiệm ca sử dụng cho các đối tượng của các lớp phân tích:}{}{Ảnh chương 3/3.2/Đóng góp/Thêm đóng góp.png}{}

    \getimg{Biểu đồ trình tự cho usecase “Xoá khoản thu tự nguyện” phân bổ trách nhiệm ca sử dụng cho các đối tượng của các lớp phân tích:}{}{Ảnh chương 3/3.2/Đóng góp/Xoá đóng góp.png}{}
\newpage
%---------------------------------------------------------

    \getimg{Biểu đồ trình tự cho usecase “Xem thông báo” phân bổ trách nhiệm ca sử dụng cho các đối tượng của các lớp phân tích:}{}{Ảnh chương 3/3.2/Thông báo/Xem thông báo.png}{}

    \getimg{Biểu đồ trình tự cho usecase “Tìm kiếm thông báo” phân bổ trách nhiệm ca sử dụng cho các đối tượng của các lớp phân tích:}{}{Ảnh chương 3/3.2/Thông báo/Tìm thông báo.png}{}

    \getimg{Biểu đồ trình tự cho usecase “Thêm thông báo” phân bổ trách nhiệm ca sử dụng cho các đối tượng của các lớp phân tích:}{}{Ảnh chương 3/3.2/Thông báo/Thêm thông báo.png}{}

    \getimg{Biểu đồ trình tự cho usecase “Xoá thông báo” phân bổ trách nhiệm ca sử dụng cho các đối tượng của các lớp phân tích:}{}{Ảnh chương 3/3.2/Thông báo/Xoá thông báo.png}{}
    
    \getimg{Biểu đồ trình tự cho usecase “Sửa thông báo” phân bổ trách nhiệm ca sử dụng cho các đối tượng của các lớp phân tích:}{}{Ảnh chương 3/3.2/Thông báo/Sửa thông báo.png}{}
\newpage
%---------------------------------------------------------

    \getimg{Biểu đồ trình tự cho usecase “Xem danh sách thu phí xe" phân bổ trách nhiệm ca sử dụng cho các đối tượng của các lớp phân tích:}{}{Ảnh chương 3/3.2/Trông xe/Xem trông xe.png}{}

    \getimg{Biểu đồ trình tự cho usecase “Tìm kiếm thu phí xe” phân bổ trách nhiệm ca sử dụng cho các đối tượng của các lớp phân tích:}{}{Ảnh chương 3/3.2/Trông xe/Tìm trông xe.png}{}

    \getimg{Biểu đồ trình tự cho usecase “Thêm thu phí xe” phân bổ trách nhiệm ca sử dụng cho các đối tượng của các lớp phân tích:}{}{Ảnh chương 3/3.2/Trông xe/Thêm trông xe.png}{}

    \getimg{Biểu đồ trình tự cho usecase “Xoá thu phí xe” phân bổ trách nhiệm ca sử dụng cho các đối tượng của các lớp phân tích:}{}{Ảnh chương 3/3.2/Trông xe/Xoá trông xe.png}{}

    \getimg{Biểu đồ trình tự cho usecase “Sửa thu phí xe” phân bổ trách nhiệm ca sử dụng cho các đối tượng của các lớp phân tích:}{}{Ảnh chương 3/3.2/Trông xe/Sửa trông xe.png}{}
\newpage

% 3.3---------------------------------------------------------
\subsection{Xây dựng biểu đồ lớp phân tích}

    \getimg{Usecase Quản lý nhân khẩu:}{}{Ảnh chương 3/3.3/Nhân khẩu usecase.png}{}

    \getimg{Usecase quản lý hộ khẩu:}{}{Ảnh chương 3/3.3/Hộ khẩu usecase.png}{}

    \getimg{Usecase quản lý khoản thu bắt buộc:}{}{Ảnh chương 3/3.3/Khoản thu usecase.png}{}

    \getimg{Usecase quản lý khoản thu tự nguyện:}{}{Ảnh chương 3/3.3/Đóng góp usecase.png}{}

    \getimg{Usecase quản lý khoản thu tự nguyện:}{}{Ảnh chương 3/3.3/Trông xe usecase.png}{}
    
    \getimg{Usecase quản lý khoản thu tự nguyện:}{}{Ảnh chương 3/3.3/Thông báo usecase.png}{}

% 3.4 ---------------------------------------------------------
\subsection{Xây dựng biểu đồ thực thể liên kết (ERD)}
- Xác định các đối tượng dữ liệu: các đối tượng dữ liệu bao gồm nhân khẩu, hộ khẩu, các khoản phí cần nộp

- Xác định các đặc tính của đối tượng dữ liệu:
\begin{itemize}[leftmargin = 1.5cm]
    \item Nhân khẩu: Mã dân cư, tên dân cư, ngày sinh, giới tính, số hộ khẩu, trạng thái, số căn cước công dân, số điện thoại, quan hệ với chủ hộ
    
    \item Hộ khẩu: Mã căn hộ (Định danh hộ khẩu), số thành viên trong hộ khẩu, số căn hộ, tên chủ hộ, số điện thoại, tên chủ hộ
    
    \item Khoản thu bắt buộc: Mã hộ dân, hạn thu, tên khoản thu, số tiền cần thu, số tiền đã thu, trạng thái

    \item Khoản thu tự nguyện: Mã hộ dân, tên khoản thu, trạng thái, số tiền đã thu

    \item Trông xe: Mã hộ dân, số ô tô, số xe máy, số tiền cần thu, số tiền đã thu, hàn thu, trạng thái
\end{itemize}

- Các mối quan hệ giữa các đối tượng dữ liệu : 
\begin{itemize}[leftmargin = 1.5cm]
    \item Hộ khẩu sẽ chứa nhiều nhân khẩu hay 1 nhân khẩu sẽ thuộc (nằm trong) 1 hộ khẩu.
    \item Nhân khẩu là chủ hộ của hộ khẩu.
    \item Khoản thu bắt buộc là sự hợp thành từ 1 khoản thu và 1 hộ khẩu buộc phải trả.
    \item Khoản thu tự nguyện là sự kết hợp từ 1 khoản thu và 1 hộ khẩu góp tự nguyện.
    \item Để có thể được trông xe thì phải đăng ký, 1 hộ khẩu có thể đăng ký cho nhiều xe
    \item Sau khi đăng ký trông xe, khoản phí gửi xe trở thành bắt buộc, 1 dân cư có thể đại diện 1 hộ dân trả tiền cho các khoản phí bắt buộc định kỳ
\end{itemize}

- Biểu đồ ERD mô tả mối quan hệ giữa các đối tượng dữ liệu:
    \getimg{}{}{Ảnh chương 3/3.4/ERD.png}{}
\newpage 

%---------------------------------------------------------------------
\section*{CHƯƠNG 4. THIẾT KẾ CHƯƠNG TRÌNH}
\phantomsection % Đảm bảo liên kết chính xác
\addcontentsline{toc}{section}{CHƯƠNG 4. THIẾT KẾ CHƯƠNG TRÌNH}
\setcounter{section}{4}
\setcounter{subsection}{0}

\subsection{Thiết kế kiến trúc}
Phần mềm phát triển dựa trên kiến trúc MVC. Mẫu kiến trúc MVC là phương
pháp chia nhỏ các thành phần dữ liệu, trình bày và dữ liệu nhập từ người dùng thành
những thành phần riêng biệt.

Từ sơ đồ kiến trúc MVC chung, nhóm đã xây dựng và phát triển phần mềm dựa
trên khung của sơ đồ kiến trúc này. Cụ thể, thành phần Model trong phần mềm là bao
gồm gói model và service, model định nghĩa và khởi tạo ra các đối tượng cần thiết phù
họp với những dữ liệu trong cơ sở dữ liệu, service cung cấp các thao tác trực tiếp tới
cơ sở dữ liệu để có thể dễ dàng thêm, xóa, sửa dễ hơn trên cơ sở dữ liệu. Các thành phần thiết kế giao diện được tạo từ các file html, css và js. Thành phần Controller là các class controller, để điều khiển các thao tác từ người dùng.

\subsection{Thiết kế cơ sở dữ liệu}
Sơ đồ quan hệ giữa các bảng:
\begin{figure}[H]
    \centering
    \includegraphics[width=\linewidth]{Ảnh chương 4/4.2/Database.png}
\end{figure}

% \describetable{Đặc tả dữ liệu cho bảng hộ khẩu:}{
%     \addRow\getRowWithColor{\textbf{\underline{household\_id}}}{bigint(20)}{20 chữ số}{Khoá chính}{Số nguyên dương}{}{white}
%     \addRow\getRowWithColor{apartment\_size}{double}{60 ký tự}{not null}{Số thực dương}{}{lightblue}
%     \addRow\getRowWithColor{household \_number}{varchar(50)}{50 ký tự}{not null}{Văn bản}{}{white}
%     \addRow{\getRowWithColor{\textbf{\underline{owner\_id}}}{bigint(20)}{20 chữ số}{Khoá tham chiếu bảng resident}{Số nguyên dương}{}{lightblue}}
% }

% \describetable{Đặc tả dữ liệu cho bảng tài khoản:}
% {
%     \addRow\getRowWithColor{\textbf{\underline{id}}}{bigint(20)}{20 chữ số}{Khoá chính}{Số nguyên dương}{}{white}
%     \addRow\getRowWithColor{username}{varchar(255)}{255 ký tự}{not null}{Văn bản}{}{lightblue}

%     \addRow\getRowWithColor{password}{varchar(255)}{255 ký tự}{Khoá chính}{Số nguyên dương}{}{white}
%     \addRow\getRowWithColor{role}{Varchar(255)}{255 ký tự}{not null}{Văn bản}{}{lightblue}

%     \addRow\getRowWithColor{\textbf{\underline{resident\_id}}}{bigint(20)}{20 chữ số}{Khoá tham chiếu bảng resident}{Sô nguyên dương}{}{white}
% }

% \describetable{Đặc tả dữ liệu cho bảng nhân khẩu:}
% {
%     \addRow\getRowWithColor{\textbf{\underline{resident\_id}}}{bigint(20)}{20 chữ số}{Khoá chính}{Số nguyên dương}{}{white}
%     \addRow\getRowWithColor{date\_of\_birth}{date}{}{not null}{Ngày tháng năm}{}{lightblue}
    
%     \addRow\getRowWithColor{sex}{varchar(255)}{20 chữ số}{not null}{Số nguyên dương}{}{white}
%     \addRow\getRowWithColor{id\_card}{varchar(20)}{20 ký tự}{not null}{}{}{lightblue}
    
%     \addRow\getRowWithColor{name}{varchar(30)}{30 ký tự}{not null}{Số nguyên \newline dương}{}{white}
%     \addRow\getRowWithColor{phone}{varchar(255)}{255 ký tự}{not null}{Số điện thoại}{}{lightblue}
    
%     \addRow\getRowWithColor{quan he voi chu ho}{varchar(255)}{255 ký tự}{not null}{Văn bản}{}{white}
%     \addRow\getRowWithColor{temporary}{varchar(255)}{255 ký tự}{not null}{Văn bản}{}{lightblue}
    
%     \addRow\getRowWithColor{\textbf{\underline{household\_id}}}{bigint(20)}{20 chữ số}{Khoá tham chiếu bảng household}{Số nguyên dương}{}{white}
% }

% \describetable{Đặc tả dữ liệu cho bảng khoản phí bắt buộc:}
% {
%     \addRow\getRowWithColor{\textbf{\underline{contribution\_id}}}{bigint(20)}{20 chữ số}{Khoá chính}{Số nguyên dương}{}{white}
%     \addRow\getRowWithColor{amount}{double}{60 ký tự}{not null}{Văn bản}{}{lightblue}

%     \addRow\getRowWithColor{contribution \_type}{varchar(50)}{50 ký tự}{not null}{Số nguyên dương}{}{white}
%     \addRow\getRowWithColor{date\_contributed}{date}{60 ký tự}{not null}{Văn bản}{}{lightblue}

%     \addRow\getRowWithColor{\textbf{\underline{household\_id}}}{bigint(20)}{20 chữ số}{Khoá tham chiếu bảng household}{Số nguyên dương}{}{white}
% }

% \describetable{Đặc tả dữ liệu cho bảng khoản phí tự nguyện:}
% {
%     \addRow\getRowWithColor{\textbf{\underline{fee\_id}}}{bigint(20)}{20 chữ số}{Khoá chính}{Số nguyên dương}{}{white}
%     \addRow\getRowWithColor{amount}{double}{60 ký tự}{not null}{Văn bản}{}{lightblue}

%     \addRow\getRowWithColor{due\_date}{bigint(20)}{}{not null}{Số nguyên dương}{}{white}
%     \addRow\getRowWithColor{fee\_type}{varchar(50)}{50 ký tự}{not null}{Văn bản}{}{lightblue}

%     \addRow\getRowWithColor{paid}{bit(1)}{}{not null}{boolean}{}{white}
%     \addRow\getRowWithColor{\textbf{\underline{household\_id}}}{bigint(20)}{20 chữ số}{Khoá tham chiếu bảng household}{Số nguyên dương}{}{lightblue}
% }
\newpage

% 4.3-----------------------------------------------------------------------
\subsection{Thiết kế chi tiết các gói}

\getimg{Biểu đồ package cho gói Controller:}{0.9}{Ảnh chương 4/4.3/Controller.png}{}
\getimg{Biểu đồ package cho gói Model:}{0.8}{Ảnh chương 4/4.3/Model.png}{}
\getimg{Biểu đồ package cho gói Service:}{0.9}{Ảnh chương 4/4.3/Service.png}{}
\getimg{Biểu đồ package cho gói Repository:}{0.8}{Ảnh chương 4/4.3/Repository.png}{}
\getimg{Biểu đồ package cho gói Exception:}{0.7}{Ảnh chương 4/4.3/Exception.png}{}
\getimg{Biểu đồ package cho gói DTO:}{0.7}{Ảnh chương 4/4.3/DTO.png}{}
\getimg{Biểu đồ package cho gói Config:}{0.7}{Ảnh chương 4/4.3/Config.png}{}
\newpage

% % 4.4 -----------------------------------------------------------------
\subsection{Thiết kế chi tiết lớp}
\getimg{Class Household:}{0.5}{Ảnh chương 4/4.4/Class/Household.png}{}
\getimg{Class Users:}{0.5}{Ảnh chương 4/4.4/Class/Users.png}{}
\getimg{Class Resident:}{0.45}{Ảnh chương 4/4.4/Class/Resident.png}{}
\getimg{Class Contribution:}{0.45}{Ảnh chương 4/4.4/Class/Contribution.png}{}
\getimg{Class Fee:}{0.5}{Ảnh chương 4/4.4/Class/Fee.png}{}
\getimg{Class Notification:}{0.5}{Ảnh chương 4/4.4/Class/Notification.png}{}
\newpage

% 4.5 ---------------------------------------------------------------------
\subsection{Sơ đồ lớp chi tiết}

\getimg{Lớp liên quan đến chức năng nhân khẩu:}{}{Ảnh chương 4/4.5/Resident_DCD.png}{}

\getimg{Lớp liên quan đến chức năng hộ khẩu:}{}{Ảnh chương 4/4.5/Household_DCD.png}{}

\getimg{Lớp liên quan đến chức năng khoản thu bắt buộc:}{}{Ảnh chương 4/4.5/Fee_DCD.png}{}

\getimg{Lớp liên quan đến chức năng khoản thu tự nguyện:}{}{Ảnh chương 4/4.5/Contribute_DCD.png}{}

\getimg{Lớp liên quan đến chức năng khoản thu phí giữ xe:}{}{Ảnh chương 4/4.5/ParkingFee_DCD.png}{}

\getimg{Lớp liên quan đến chức năng thông báo:}{}{Ảnh chương 4/4.5/Notification_DCD.png}{}

\newpage

% % 4.6 ---------------------------------------------------------------------
\subsection{Thiết kế giao diện}
\textbf{Thiết kế mock-up cho từng giao diện của bài toán}

% Hộ khẩu
\getimg{Mock-up cho giao diện đăng nhập của bài toán:}{0.8}{Ảnh chương 4/Đăng nhập.png}{}
\getimg{Mock-up cho màn hình chính của bài toán:}{0.8}{Ảnh chương 4/Màn hình chính.png}{}
\getimg{Mock-up cho màn hình hộ khẩu của bài toán (Admin):}{0.8}{Ảnh chương 4/Màn hình hộ khẩu.png}{}
\getimg{Mock-up cho màn hình hộ khẩu của bài toán (User):}{0.8}{Ảnh chương 4/Màn hình hộ khẩu (cư dân).png}{}
\getimg{Mock-up cho màn hình xem thông tin chi tiết hộ khẩu của bài toán (Admin):}{0.75}{Ảnh chương 4/Màn hình thông tin hộ khẩu (admin).png}{}
\getimg{Mock-up cho màn hình thêm hộ khẩu của bài toán (Admin):}{0.75}{Ảnh chương 4/Màn hình thêm hộ khẩu.png}{}
\getimg{Mock-up cho màn hình chuyển hộ khẩu của bài toán (Admin):}{0.9}{Ảnh chương 4/Màn hình chuyển hộ khẩu.png}{}

% Dân cư
\getimg{Mock-up cho màn hình nhân khẩu của bài toán (Admin):}{0.9}{Ảnh chương 4/Màn hình dân cư.png}{}
\getimg{Mock-up cho màn hình đăng ký nhân khẩu của bài toán (Admin):}{0.8}{Ảnh chương 4/Màn hình thêm thông tin dân cư.png}{}
\getimg{Mock-up cho màn hình thông tin dân cư của bài toán (Admin):}{0.8}{Ảnh chương 4/Màn hình thông tin cư dân (admin).png}{}
\getimg{Mock-up cho màn hình thông tin dân cư của bài toán (User):}{0.9}{Ảnh chương 4/Màn hình thông tin cư dân.png}{}

% Phí bắt buộc
\getimg{Mock-up cho màn hình thu phí của bài toán (Admin):}{0.9}{Ảnh chương 4/Màn hình thu phí.png}{}
\getimg{Mock-up cho màn hình thu phí của bài toán (User):}{0.9}{Ảnh chương 4/Màn hình khoản thu (cư dân).png}{}

% Phí tự nguyện
\getimg{Mock-up cho màn hình khoản đóng góp tự nguyện của bài toán (Admin):}{0.9}{Ảnh chương 4/Màn hình đóng góp tình nguyện.png}{}
\getimg{Mock-up cho màn hình khoản đóng góp tự nguyện của bài toán (User):}{0.9}{Ảnh chương 4/Màn hình đóng góp (cư dân).png}{}

% Quản lý tài khoản
\getimg{Mock-up cho màn hình quản lý tài khoản (Admin):}{0.9}{Ảnh chương 4/Màn hình quản lý tài khoản.png}{}

% Màn hình cái đặt
\getimg{Mock-up cho màn hình cài đặt (Admin):}{0.9}{Ảnh chương 4/Màn hình cài đặt.png}{}
\getimg{Mock-up cho màn hình cài đặt (User):}{0.9}{Ảnh chương 4/Màn hình cài đặt User.png}{}

\textbf{Đặc tả thiết kế giao diện cho các màn hình}

\getimg{Giao diện đăng nhập của bài toán:}{0.8}{Ảnh chương 4/Đăng nhập 1.png}{}
\getimg{Màn hình chính của bài toán:}{0.8}{Ảnh chương 4/Màn hình chính 1.png}{}
\getimg{Màn hình hộ khẩu của bài toán (Admin):}{0.8}{Ảnh chương 4/Hộ khẩu Admin 1.png}{}
\getimg{}{0.8}{Ảnh chương 4/Hộ khẩu Admin 2.png}{}
\getimg{Màn hình hộ khẩu của bài toán (User):}{0.8}{Ảnh chương 4/Hộ khẩu User 1.png}{}
\getimg{Màn hình xem thông tin chi tiết hộ khẩu của bài toán (Admin):}{0.9}{Ảnh chương 4/Thông tin hộ khẩu Admin 1.png}{}
\getimg{Màn hình thêm hộ khẩu của bài toán (Admin):}{0.9}{Ảnh chương 4/Thêm hộ khẩu 1.png}{}
\getimg{Màn hình chuyển hộ khẩu của bài toán (Admin):}{0.9}{Ảnh chương 4/Chuyẻn hộ khẩu 1.png}{}
\getimg{Màn hình nhân khẩu của bài toán (Admin):}{0.9}{Ảnh chương 4/Dân cư Admin 1.png}{}
\getimg{}{0.9}{Ảnh chương 4/Dân cư Admin 2.png}{}
\getimg{Màn hình đăng ký nhân khẩu của bài toán (Admin):}{0.9}{Ảnh chương 4/Đăng ký nhân khẩu 1.png}{}
\getimg{Màn hình thông tin dân cư của bài toán (Admin):}{0.9}{Ảnh chương 4/Thông tin dân cư 1.png}{}
\getimg{Màn hình thông tin dân cư của bài toán (User):}{0.9}{Ảnh chương 4/Thông tin dân cư User 1.png}{}
\getimg{Màn hình thu phí của bài toán (Admin):}{0.9}{Ảnh chương 4/Thu phí Admin 1.png}{}
\getimg{}{0.9}{Ảnh chương 4/Thông tin thu phí Admin 2.png}{}
\getimg{Màn hình thu phí của bài toán (User):}{0.9}{Ảnh chương 4/Khoản thu User 1.png}{}
\getimg{Màn hình quản lý tài khoản (Admin):}{0.75}{Ảnh chương 4/Quản lý tài khoản Admin 1.png}{}
\getimg{Màn hình sửa đổi thông tin tài khoản (Admin):}{0.75}{Ảnh chương 4/Sửa đổi tài khoản 1.png}{}
\getimg{Màn hình cài đặt (Admin):}{0.9}{Ảnh chương 4/Đổi mật khẩu Admin 1.png}{}
\getimg{Màn hình cài đặt (User):}{0.9}{Ảnh chương 4/Đổi mật khẩu User 1.png}{}
\newpage

%---------------------------------------------------------------------
\section*{CHƯƠNG 5. XÂY DỰNG CHƯƠNG TRÌNH MINH HỌA}
\phantomsection % Đảm bảo liên kết chính xác
\addcontentsline{toc}{section}{CHƯƠNG 5. XÂY DỰNG CHƯƠNG TRÌNH MINH HỌA}
\setcounter{section}{5}
\setcounter{subsection}{0}
\subsection{Thư viện và công cụ sử dụng}
\begin{center}
    \textbf{Danh sách thư viện và công cụ sử dụng} \\
    \begin{tabular}{|c|c|c|}
        \hline
        \textbf{Mục đích} & \textbf{Công cụ} & \textbf{Địa chỉ URL} \\
        \hline
        IDE Lập trình & IntelliJ IDEA & https://lp.jetbrains.com\\
        \hline
        Thư viện & Java Spring Boot & https://spspring.io\\
        \cline{2-3}
         & Java Scripts & https://www.javascript.com/\\
         \cline{2-3}
         & HTML, CSS & \\
        \hline
    \end{tabular}
\end{center}
-----------------------------------------------------------------------------------------
\subsection{Kết quả chương trình minh họa}
Chương trình quản lý dân cư và thu phí là một giải pháp số hóa, hỗ trợ các tổ chức và cơ quan quản lý thực hiện hiệu quả các nhiệm vụ như:

\begin{itemize}
    \item Theo dõi thông tin cư dân.
    \item Quản lý tài khoản người dùng.
    \item Xử lý và cập nhật dữ liệu tài chính liên quan đến việc thu phí dịch vụ.
\end{itemize}

\textbf{Mục tiêu của chương trình}
\begin{enumerate}
    \item \textbf{Tự động hóa quy trình}:
        Giảm thiểu khối lượng công việc thủ công thông qua việc ứng dụng công nghệ, tiết kiệm thời gian và công sức.
    \item \textbf{Đảm bảo minh bạch và chính xác}:
        Các dữ liệu cư dân, khoản phí và lịch sử giao dịch đều được lưu trữ rõ ràng, dễ dàng truy xuất và kiểm tra.
    \item \textbf{Tăng trải nghiệm người dùng}:
        Cư dân có thể dễ dàng tra cứu thông tin tài khoản, thanh toán và cập nhật hồ sơ một cách nhanh chóng, chính xác.
\end{enumerate}

\textbf{Chức năng nổi bật}
\begin{enumerate}
    \item \textbf{Quản lý tài khoản cư dân}:
        Cung cấp giao diện đơn giản để tạo, sửa, xóa thông tin người dùng.
        Phân quyền tài khoản (ADMIN, USER) đảm bảo hệ thống an toàn và bảo mật.
    \item \textbf{Thu phí dịch vụ}:
        Ghi nhận thông tin các khoản phí dịch vụ theo từng cư dân, bao gồm phí quản lý, điện nước, và các dịch vụ khác.
    \item \textbf{Phân trang và lọc dữ liệu}:
        Hỗ trợ quản lý dữ liệu cư dân lớn nhờ tính năng phân trang và tìm kiếm theo các tiêu chí như tên, mã cư dân.
    \item \textbf{Bảo mật thông tin}:
        Sử dụng mã hóa mật khẩu với công nghệ tiên tiến (BCrypt), đảm bảo dữ liệu người dùng được bảo vệ tốt nhất.
\end{enumerate}

\textbf{Kết quả kỳ vọng}\\
Chương trình giúp nâng cao hiệu quả quản lý, tối ưu hóa quy trình thu phí và mang lại sự hài lòng cho cư dân cũng như đội ngũ quản lý. Đây là một bước tiến quan trọng hướng tới quản lý thông minh và bền vững trong thời đại công nghệ số.

\subsection{Giao diện minh hoạ các chức năng của chương trình}
Thiết kế giao diện thực tế cho bài toán:
\begin{itemize}
    \item Màn hình giao diện đăng nhập của bài toán :
    \begin{figure}[H]
        \centering
        \includegraphics[width=1\textwidth]{Ảnh chương 4/Login.png}
    \end{figure}
    \vspace{7cm}
    \item Màn hình chính của bài toán :
    \begin{figure}[H]
        \centering
        \includegraphics[width=1\textwidth]{Ảnh chương 4/Home Admin.png}
    \end{figure}
    \vspace{1cm}
    \item Màn hình hộ khẩu của bài toán (Admin):
    \begin{figure}[H]
        \centering
        \includegraphics[width=1\textwidth]{Ảnh chương 4/Thông tin hộ khẩu Admin.png}
    \end{figure}
    \vspace{4cm}
    \item Màn hình hộ khẩu của bài toán (User):
    \begin{figure}[H]
        \centering
        \includegraphics[width=1\textwidth]{Ảnh chương 4/Thông tin hộ khẩu User.png}
    \end{figure}
    \vspace{1cm}
    \item Màn hình xem thông tin chi tiết hộ khẩu của bài toán (Admin):
    \begin{figure}[H]
        \centering
        \includegraphics[width=1\textwidth]{Ảnh chương 4/Thông tin hộ khẩu Admin.png}
    \end{figure}
    \vspace{5cm}
    \item Màn hình thêm hộ khẩu của bài toán (Admin):
    \begin{figure}[H]
        \centering
        \includegraphics[width=0.8\textwidth]{Ảnh chương 4/Thêm hộ khẩu Admin.png}
    \end{figure}
    \item Màn hình chuyển hộ khẩu của bài toán (Admin):
    \begin{figure}[H]
        \centering
        \includegraphics[width=0.9\textwidth]{Ảnh chương 4/Chuyển hộ khẩu Admin.png}
    \end{figure}
    \item Màn hình nhân khẩu của bài toán (Admin):
    \begin{figure}[H]
        \centering
        \includegraphics[width=0.8\textwidth]{Ảnh chương 4/Admin dân cư.png}
    \end{figure}
    \item Màn hình đăng ký nhân khẩu của bài toán (Admin):
    \begin{figure}[H]
        \centering
        \includegraphics[width=1\textwidth]{Ảnh chương 4/Thêm cư dân admin.png}
    \end{figure}
    \vspace{1cm}
    \item Màn hình thông tin dân cư của bài toán (Admin):
    \begin{figure}[H]
        \centering
        \includegraphics[width=1\textwidth]{Ảnh chương 4/Thông tin cư dân Admin.png}
    \end{figure}
    \vspace{4cm}
    \item Màn hình thông tin dân cư của bài toán (User):
    \begin{figure}[H]
        \centering
        \includegraphics[width=1\textwidth]{Ảnh chương 4/Thông tin cư dân User.png}
    \end{figure}
    \vspace{2cm}
    \item Màn hình thu phí của bài toán (Admin) :
    \begin{figure}[H]
        \centering
        \includegraphics[width=1\textwidth]{Ảnh chương 4/Thông tin thu phí Admin.png}
    \end{figure}
    \vspace{3cm}
    \item Màn hình thu phí của bài toán (User) :
    \begin{figure}[H]
        \centering
        \includegraphics[width=1\textwidth]{Ảnh chương 4/Thông tin thu phí User.png}
    \end{figure}
    \vspace{2cm}
    \item Màn hình khoản đóng góp tự nguyện của bài toán (Admin) :
    \begin{figure}[H]
        \centering
        \includegraphics[width=1\textwidth]{Ảnh chương 4/Tình nguyện Admin.png}
    \end{figure}
    \vspace{2cm}
    \item Màn hình khoản đóng góp tự nguyện của bài toán (User) :
    \begin{figure}[H]
        \centering
        \includegraphics[width=1\textwidth]{Ảnh chương 4/Tình nguyện User.png}
    \end{figure}
    \vspace{2cm}
    \item Màn hình quản lý tài khoản (Admin) :
    \begin{figure}[H]
        \centering
        \includegraphics[width=1\textwidth]{Ảnh chương 4/Quản lý tài khoản Admin.png}
    \end{figure}
    \vspace{2cm}
    \item Màn hình sửa đổi thông tin tài khoản (Admin) :
    \begin{figure}[H]
        \centering
        \includegraphics[width=1\textwidth]{Ảnh chương 4/Sửa đổi tài khoản Admin.png}
    \end{figure}
    \vspace{1cm}
    \item Màn hình cài đặt (Admin):
    \begin{figure}[H]
        \centering
        \includegraphics[width=1\textwidth]{Ảnh chương 4/Màn hình đổi mật khẩu Admin.png}
    \end{figure}
    \vspace{3cm}
    \item Màn hình cài đặt (User):
    \begin{figure}[H]
        \centering
        \includegraphics[width=1\textwidth]{Ảnh chương 4/Màn hình đổi mật khẩu User.png}
    \end{figure}
\end{itemize}
\newpage

%---------------------------------------------------------------------
\section*{CHƯƠNG 6. KIỂM THỬ CHƯƠNG TRÌNH}
\addcontentsline{toc}{section}{CHƯƠNG 6. KIỂM THỬ CHƯƠNG TRÌNH}
\setcounter{section}{6}
\setcounter{subsection}{0}
\subsection{Kiểm thử các chức năng đã thực hiện}
\subsubsection{Kiểm thử cho chức năng đăng nhập}
\begin{itemize}
    \item Chức năng đăng nhập :
    \begin{figure}[H]
        \centering
    \includegraphics[width=1\textwidth]{Kiểm thử/Kiểm thử đăng nhập.png}
    \end{figure}
\end{itemize}
\subsubsection{Kiểm thử cho chức năng quản lý nhân khẩu}
\begin{itemize}
    \item Chức năng thêm mới nhân khẩu
    \begin{figure}[H]
        \centering
        \includegraphics[width=1\textwidth]{Kiểm thử/Kiểm thử thêm nhân khẩu.png}
    \end{figure}
    \item Chức năng xóa nhân khẩu:
    \begin{figure}[H]
        \centering
        \includegraphics[width=1\textwidth]{Kiểm thử/KT xóa nhân khẩu.png}
    \end{figure}
    \newpage
    \item Chức năng sửa nhân khẩu:
    \begin{figure}[H]
        \centering
        \includegraphics[width=1\textwidth]{Kiểm thử/KT sửa nhân khẩu.png}
    \end{figure}
    \item Chức năng tìm kiếm nhân khẩu:
    \begin{figure}[H]
        \centering
        \includegraphics[width=1\textwidth]{Kiểm thử/KT tìm hộ khẩu.png}
    \end{figure}
    \item Chức năng xem thông tin nhân khẩu:
    \begin{figure}[H]
        \centering
        \includegraphics[width=1\textwidth]{Kiểm thử/KT hiển thị thông tin chi tiết nhân khẩu.png}
    \end{figure}
\end{itemize}
\newpage
\subsubsection{Kiểm thử cho chức năng quản lý hộ khẩu}
\begin{itemize}
    \item Chức năng thêm mới hộ khẩu
    \begin{figure}[H]
        \centering
        \includegraphics[width=1\textwidth]{Kiểm thử/KT thêm hộ khẩu.png}
    \end{figure}
    \item Chức năng xóa hộ khẩu:
    \begin{figure}[H]
        \centering
        \includegraphics[width=1\textwidth]{Kiểm thử/KT xóa hộ khẩu.png}
    \end{figure}
    \item Chức năng xem thông tin hộ khẩu:
    \begin{figure}[H]
        \centering
        \includegraphics[width=1\textwidth]{Kiểm thử/KT xem thông tin hộ.png}
    \end{figure}
    \item Chức năng tìm kiếm hộ khẩu:
    \begin{figure}[H]
        \centering
        \includegraphics[width=1\textwidth]{Kiểm thử/KT tìm hộ khẩu.png}
    \end{figure}
    \newpage
    \item Chức năng chuyển chủ hộ khẩu:
    \begin{figure}[H]
        \centering
        \includegraphics[width=1\textwidth]{Kiểm thử/KT sửa chủ hộ.png}
    \end{figure}
    \item Chức năng sửa thông tin hộ khẩu:
    \begin{figure}[H]
        \centering
        \includegraphics[width=1\textwidth]{Kiểm thử/Kt sửa thông tin hộ.png}
    \end{figure}
\end{itemize}
\subsubsection{Kiểm thử cho chức năng quản lý khoản thu tự nguyện}
\begin{itemize}
    \item Chức năng tìm kiếm khoản thu:
    \begin{figure}[H]
        \centering
        \includegraphics[width=1\textwidth]{Kiểm thử/Kt tìm kiếm khoản thu.png}
    \end{figure}

    \item Chức năng xóa khoản thu:
    \begin{figure}[H]
        \centering
        \includegraphics[width=1\textwidth]{Kiểm thử/KT xóa khoản thu.png}
    \end{figure}
    \newpage
    \item Chức năng thêm khoản thu:
    \begin{figure}[H]
        \centering
        \includegraphics[width=1\textwidth]{Kiểm thử/KT thêm khoản thu.png}
    \end{figure}
    \item Chức năng xem chi tiết khoản thu:
    \begin{figure}[H]
        \centering
        \includegraphics[width=1\textwidth]{Kiểm thử/KT chi tiết khoản thu.png}
    \end{figure}    
\end{itemize}
\subsubsection{Kiểm thử cho chức năng quản lý khoản thu bắt buộc}
\begin{itemize}
    \item Chức năng tìm kiếm khoản thu:
    \begin{figure}[H]
        \centering
        \includegraphics[width=1\textwidth]{Kiểm thử/Kt tìm kiếm khoản thu.png}
    \end{figure}
    \item Chức năng xóa khoản thu:
    \begin{figure}[H]
        \centering
        \includegraphics[width=1\textwidth]{Kiểm thử/KT xóa khoản thu.png}
    \end{figure}
    \newpage
    \item Chức năng sửa khoản thu:
    \begin{figure}[H]
        \centering
        \includegraphics[width=1\textwidth]{Kiểm thử/KT sửa khoản thu.png}
    \end{figure}
    \item Chức năng thêm khoản thu:
    \begin{figure}[H]
        \centering
        \includegraphics[width=1\textwidth]{Kiểm thử/KT thêm khoản thu.png}
    \end{figure}
    \item Chức năng xem chi tiết khoản thu:
    \begin{figure}[H]
        \centering
        \includegraphics[width=1\textwidth]{Kiểm thử/KT chi tiết khoản thu.png}
    \end{figure}    
\end{itemize}
\subsection{Kiểm thử chức năng quản lý gửi xe}
\begin{itemize}
    \item Chức năng tìm kiếm khoản thu:
    \begin{figure}[H]
        \centering
        \includegraphics[width=1\textwidth]{Kiểm thử/Kt tìm kiếm khoản thu.png}
    \end{figure}
    \vspace{2cm}
    \item Chức năng xóa khoản thu:
    \begin{figure}[H]
        \centering
        \includegraphics[width=1\textwidth]{Kiểm thử/KT xóa khoản thu.png}
    \end{figure}
    \item Chức năng sửa khoản thu:
    \begin{figure}[H]
        \centering
        \includegraphics[width=1\textwidth]{Kiểm thử/KT sửa khoản thu.png}
    \end{figure}
    \item Chức năng thêm khoản thu:
    \begin{figure}[H]
        \centering
        \includegraphics[width=1\textwidth]{Kiểm thử/KT thêm phsi gửi xe.png}
    \end{figure}
    \item Chức năng xem chi tiết khoản thu:
    \begin{figure}[H]
        \centering
        \includegraphics[width=1\textwidth]{Kiểm thử/KT chi tiết khoản thu.png}
    \end{figure}    
\end{itemize}
\newpage
\subsection{Kiểm thử chức năng thông báo}
\begin{itemize}
    \item Chức năng tìm kiếm thông báo:
    \begin{figure}[H]
        \centering
        \includegraphics[width=1\textwidth]{Kiểm thử/Kt tìm kiếm khoản thu.png}
    \end{figure}
    \item Chức năng thêm thông báo:
    \begin{figure}[H]
        \centering
        \includegraphics[width=1\textwidth]{Kiểm thử/KT thêm thông báo.png}
    \end{figure}
    \item Chức năng xem chi tiết thông báo:
    \begin{figure}[H]
        \centering
        \includegraphics[width=1\textwidth]{Kiểm thử/KT chi tiết thông báo.png}
    \end{figure}    
\end{itemize}
\subsection{Kiểm thử chức năng quản lý tài khoản}
\begin{itemize}
    \item Chức năng xóa tài khoản:
    \begin{figure}[H]
        \centering
        \includegraphics[width=1\textwidth]{Kiểm thử/KT xóa khoản thu.png}
    \end{figure}
    \newpage
    \item Chức năng tìm kiếm tài khoản:
    \begin{figure}[H]
        \centering
        \includegraphics[width=1\textwidth]{Kiểm thử/Kt tìm kiếm khoản thu.png}
    \end{figure}
    
    \item Chức năng sửa tài khoản:
    \begin{figure}[H]
        \centering
        \includegraphics[width=1\textwidth]{Kiểm thử/KT sửa tài khoản.png}
    \end{figure}
    \item Chức năng thêm tài khoản:
    \begin{figure}[H]
        \centering
        \includegraphics[width=1\textwidth]{Kiểm thử/KT thêm tài khoản.png}
    \end{figure}
\end{itemize}
\newpage
\subsection{Kiểm thử chức năng cài đặt}
\begin{itemize}
    \item Chức năng đổi mật khẩu:
    \begin{figure}[H]
        \centering
        \includegraphics[width=1\textwidth]{Kiểm thử/Kiểm thử đổi mk.png}
    \end{figure}   
\end{itemize}
\subsection{Kiểm thử chức năng đăng xuất}
\begin{itemize}
    \item Chức năng đăng xuất:
    \begin{figure}[H]
        \centering
        \includegraphics[width=1\textwidth]{Kiểm thử/KT đăng xuất.png}
    \end{figure}   
\end{itemize}
\subsubsection{Kiểm thử yêu cầu phi chức năng}
\begin{itemize}
    \item Đảm bảo chỉ người dùng có quyền hợp lệ mới truy cập được tài nguyên.
    \item Kiểm tra logic phân quyền theo vai trò (Admin/User) thành công.
    \item Thời gian phản hồi phải nhỏ hơn 2 giây cho các trang chính. 
    \item Đảm bảo website hiển thị chính xác trên các trình duyệt phổ biến như Chrome, Microsoft Edge, ..
\end{itemize}

\newpage

%---------------------------------------------------------------------
\section*{CHƯƠNG 7. HƯỚNG DẪN CÀI ĐẶT VÀ SỬ DỤNG}
\phantomsection % Đảm bảo liên kết chính xác
\addcontentsline{toc}{section}{CHƯƠNG 7. HƯỚNG DẪN CÀI ĐẶT VÀ SỬ DỤNG}
\setcounter{section}{7}
\setcounter{subsection}{0}

\subsection{Hướng dẫn cài đặt}
    \begin{itemize}
        \item Để sử dụng phần mềm, cần chuẩn bị môi trường chạy Java Spring Boot.
        \item Cài đặt một số công cụ và thư viện cần thiết như:
        \begin{itemize}
            \item \textbf{JDK} (Java Development Kit) phiên bản 11 trở lên.
            \item \textbf{Maven} để quản lý dự án và thư viện.
            \item \textbf{MySQL Server} để quản lý cơ sở dữ liệu.
            \item \textbf{Thư viện MySQL Connector} cho kết nối cơ sở dữ liệu.
            \item \textbf{Thư viện Frontend} như Bootstrap hoặc các framework JS nếu cần.
        \end{itemize}
        \item Cài đặt IDE hỗ trợ phát triển Java (IntelliJ IDEA, Eclipse, hoặc Visual Studio Code).
    \end{itemize}

\subsection{Đối tượng, phạm vi sử dụng}
    \begin{itemize} 
        \item \textbf{Đối tượng sử dụng}: Nhân viên quản lý, kế toán hoặc quản trị viên cần theo dõi thông tin cư dân và thực hiện thu phí dịch vụ hoặc cư dân trong chung cư để nắm được thông báo của khu dân cư.
        \item \textbf{Phạm vi sử dụng}: Hệ thống này được thiết kế để quản lý các khoản thu phí và thông tin cư dân tại các khu vực hoặc khu dân cư lớn. 
    \end{itemize}
    
\subsection{Xác định các yêu cầu cài đặt}
    \subsubsection{\textbf{Yêu cầu phần mềm}:}
    \begin{itemize}
        \item Java JDK 11 hoặc cao hơn.
        \item MySQL Server phiên bản 8.0 trở lên.
        \item Maven để quản lý các thư viện.
        \item Spring Boot Framework (được cấu hình trong dự án).
    \end{itemize}
    \subsubsection{\textbf{Yêu cầu phần cứng}:}
    \begin{itemize}
        \item Dung lượng ổ cứng còn trống: ít nhất 500MB để cài đặt phần mềm và lưu trữ dữ liệu.
    \end{itemize}
    
\subsection{Hướng dẫn chi tiết các bước cài đặt}
    \begin{enumerate}
        \item \textbf{Cài đặt Java JDK và thiết lập biến môi trường:}
        \begin{itemize}
            \item Tải JDK từ trang chủ: \href{https://www.oracle.com/java/technologies/javase-jdk11-downloads.html}{https://www.oracle.com/java/technologies/javase-jdk11-downloads.html}.
            \item Sau khi cài đặt, cấu hình biến môi trường \verb|JAVA_HOME| và thêm vào \verb|PATH|.
        \end{itemize}
        \item \textbf{Cài đặt MySQL Server:}
        \begin{itemize}
            \item Tải MySQL Server từ \href{https://dev.mysql.com/downloads/mysql/}{https://dev.mysql.com/downloads/mysql/}.
            \item Sau khi cài đặt, sử dụng MySQL Workbench để tạo cơ sở dữ liệu với file \verb|.sql| có sẵn trong thư mục dự án.
        \end{itemize}
        \item \textbf{Cài đặt Maven:}
        \begin{itemize}
            \item Tải và cài đặt Maven từ \href{https://maven.apache.org/download.cgi}{https://maven.apache.org/download.cgi}.
            \item Cấu hình biến môi trường \verb|MAVEN_HOME| và thêm vào \verb|PATH|.
        \end{itemize}
        \item \textbf{Cấu hình cơ sở dữ liệu trong ứng dụng:}
        \begin{itemize}
            \item Mở file \verb|application.properties| hoặc \verb|application.yml| trong dự án.
            \item Chỉnh sửa các thông tin kết nối:\\
        \verb|spring.datasource.url=jdbc:mysql://localhost:3306/ten_database|
        \verb|spring.datasource.username=ten_user|\\
        \verb|spring.datasource.password=mat_khau|
        \end{itemize}
        \item \textbf{Chạy ứng dụng:}
        \begin{itemize}
            \item Sử dụng lệnh sau để khởi chạy ứng dụng:\\\verb|mvn spring-boot:run |
            \item Ứng dụng sẽ khởi chạy trên cổng mặc định (thường là \verb|http://localhost:8080|).
        \end{itemize}
        \item \textbf{Kiểm tra giao diện web:}
        \begin{itemize}
            \item Mở trình duyệt và truy cập vào URL: \verb|http://localhost:8080|.
            \item Đăng nhập bằng tài khoản quản trị viên mặc định (nếu có) hoặc tài khoản cư dân được cấp.
        \end{itemize}.
    \end{enumerate}

\subsection{Hướng dẫn sử dụng phần mềm}
Phần mềm dùng cho kế toán để quản lý thông tin thu phí trong khu vực và dùng cho cư dân xem thông tin liên quan.

Phần mềm có 4 chức năng chính là quản lý nhân khẩu, quản lý hộ khẩu, quản lý khoản phí và quản lý tài khoản. Mỗi chức năng quản lý nhân khẩu, hộ khẩu, tài khoản đều có những chức năng con thêm, sửa, xóa, tìm kiếm thông tin trừ chức năng quản lý khoản phí không có chức năng xóa.

Để sử dụng chức năng nào nhấn trực tiếp vào chức năng đó và sử dụng
\newpage

%---------------------------------------------------------------------
\section*{KẾT LUẬN VÀ HƯỚNG PHÁT TRIỂN}
\phantomsection % Đảm bảo liên kết chính xác
\addcontentsline{toc}{section}{KẾT LUẬN VÀ HƯỚNG PHÁT TRIỂN}
Kết thúc quá trình phát triển phần mềm, đa số đã hoàn thành được những yêu cầu đã đặt ra trước đó của nhóm như là giúp xây dựng một phần mềm quản lý thu phí đơn giản, dễ sử dụng, công khai và minh bạch các khoản phí. Những chức năng quản lý nhân khẩu, hộ khẩu, khoản thu, nộp tiền đều hỗ trợ thêm, sửa, xóa, tìm kiếm thông tin, các chức năng đều dễ sử dụng.

Tuy nhiên, do thời gian có hạn nên trong quá trình phát triển cũng còn 1 số phần mà chưa được hợp lý mà chưa thể sửa chữa ngay. Trong phần quản lý các khoản thu, các khoản thu là tự nguyện nhưng số tiền nộp vẫn bị thiết lập mặc định, chưa có thống kê chi tiết về số hộ nộp các khoản phí, số hộ chưa nộp để dễ dàng trong việc quản lý. Ngoài ra, còn một số lỗi nho nhỏ khác mà nhóm có thể chưa phát hiện ra.
Phần mềm nếu hoạt động trên các cơ sở dữ liệu lớn thì sẽ bị chậm.

Trong tương lai, nhóm chúng em sẽ cố gắng hoàn thiện phát triển phần mềm để mang lại một phần mềm có trải nghiệm tốt hơn, khắc phục được những nhược điểm bên trên. Nếu có điều kiện cho phép về thời gian, nhân lực nhóm có thể phát triển phần mềm thêm nhiều chức năng khác để giúp đơn giản hóa các công việc được thực hiện thủ công rất mệt mỏi và dễ bị nhầm lẫn.
\newpage

%---------------------------------------------------------------------
\section*{TÀI LIỆU THAM KHẢO}
\phantomsection % Đảm bảo liên kết chính xác
\addcontentsline{toc}{section}{TÀI LIỆU THAM KHẢO}
\newpage

% %---------------------------------------------------------------------
\section*{PHỤ LỤC}
\phantomsection % Đảm bảo liên kết chính xác
\addcontentsline{toc}{section}{PHỤ LỤC}

\end{document}
